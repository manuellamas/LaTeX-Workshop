\documentclass[10pt]{article}
\usepackage[utf8]{inputenc}
\usepackage[portuguese]{babel}
\usepackage{eurosym} % Euro symbol

\usepackage{hyperref} % Hyperlinks
\usepackage{tikz} % Plots, Graphs, ...

% \usepackage[demo]{graphicx}
% \usepackage{subcaption}

\usepackage{listings} % to write code

\usepackage{float} % For figure placement option H


\usepackage{amsmath} % math. function by cases
\usepackage{amsfonts} % For for example \mathbb{}

\usepackage[left=2.5cm, right=3cm]{geometry}

% Macros
\newcommand{\N}{\mathbb{N}}

\title{Workshop \LaTeX Cheat Sheet}
\author{João Dionísio e Manuel Lamas}
\date{December 2022}


\begin{document}

\maketitle

\section{Introduction}

Para começar a utilizar LaTeX, \href{https://www.texstudio.org/}{TeXstudio} e \href{https://miktex.org/}{MiKTeX} são as ferramentas que aconselhamos.

O que é \LaTeX?
\begin{itemize}
    \item Diferencia-se de Word e Libre Office por não ser WYSIWYG (What you see is what you get), e por isso temos um ficheiro source code que não representa o resultado final. Tendo de visualizar o PDF separadamente.
    \item Várias funcionalidades podem ser adicionadas via packages que podem indo ser instaladas à medida que a necessidade aparece
\end{itemize}


Topics
\begin{itemize}
    \item Math expressions $ 1 + 1 = 2$
    \item Enfase através de \textbf{bold} e \emph{itálico}. % Há duas opções para itálico https://tex.stackexchange.com/a/26934
    \item Itemize and enumerate (with second and third levels as well)
    \item Tabelas
    \item Figuras
    \begin{itemize}
        \item Inserir imagens
        \item TikZ \verb|\usepackage{tikz}| \href{https://tikz.dev/tikz-plots}{TikZ Plots}
            \item Criar plots simples E.g. $y=x^{2}$ e $y=x$
            \item Um grafo simples também
        \item Posicionamento h,t,b,!,...
        \item Definir diretório para ir buscar as imagens (em vez da root, pasta "principal")
        \item Subfigures (ter figuras dispostas na vertical ou horizontal \verb|~|)
        \item Captions
        \item Referenciar figuras (\verb|\ref|) através de uma key acrescentada através do comando \verb|\label| (que tem de estar sempre depois da caption)
    \end{itemize}
    \item Comentários são escritos entre \% \verb|% Um comentário %| % Like this you see... %
    \item Escape a carateres especiais. "10\% das pessoas"
    \item Reparar que as áspas não foram colocadas corretamente. Para tal ``escrevemos assim''
    \item Links package \verb|hyperref| \href{https://en.wikibooks.org/wiki/LaTeX/Hyperlinks}{mais informação}
    \item (Mini) Exemplo de Beamer
\end{itemize}


\paragraph*{TikZ example} O pacote \verb|tikz| permite-nos criar gráficos de funções como o da figura~\ref{fig:tikz_example}. Mas serve para criar grafos, e muitos outros tipos de diagramas.
Ver mais na página \href{https://tikz.dev/}{Ti$K$Z}.

\begin{figure}[H]
    \centering
    \begin{tikzpicture}[domain=0:4]
      \draw[very thin,color=gray] (-0.1,-1.1) grid (3.9,3.9);
    
      \draw[->] (-0.2,0) -- (4.2,0) node[right] {$x$};
      \draw[->] (0,-1.2) -- (0,4.2) node[above] {$f(x)$};
    
      \draw[color=red]    plot (\x,\x)             node[right] {$f(x) =x$};
      % \x r means to convert '\x' from degrees to _r_adians:
      \draw[color=blue]   plot (\x,{sin(\x r)})    node[right] {$f(x) = \sin x$};
      \draw[color=orange] plot (\x,{0.05*exp(\x)}) node[right] {$f(x) = \frac{1}{20} \mathrm e^x$};
    \end{tikzpicture}
    \caption{Caption}
    \label{fig:tikz_example}
\end{figure}


\paragraph*{Extras:}
\href{https://www.overleaf.com/}{OverLeaf} é uma boa ferramente para trabalho colaborativo (particularmente em tempo real).


\paragraph*{Macros:}
Podem-se criar comandos para facilitar o uso daqueles que se usam com maior frequência. Por exemplo o comando \verb|\N| pode ser criado para replicar $\mathbb{N}$ (de \verb|\mathbb{N}|).

E então $\N$.

O macro é criado acrescentando a seguinte linha ao preâmbulo:\\
\verb|\newcommand{\N}{\mathbb{N}}|

% Este exemplo é mais ilustrativo, normalmente criariamos macros não cópias de um comando.
% Mas de comandos mais compridos e detalhados que deixassem de tornassem aborrecido o seu uso

\end{document}