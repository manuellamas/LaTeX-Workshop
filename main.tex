\documentclass[10pt]{article}
\usepackage[utf8]{inputenc}

%\interfootnotelinepenalty=10000 %force footnote in the same page
\usepackage{eurosym} % Euro symbol

\usepackage{hyperref} % Hyperlinks
\usepackage{tikz} %

\usepackage[demo]{graphicx}
\usepackage{subcaption}
\usepackage{tabularx}


\usepackage{float} % to force image placement

\usepackage{listings} % to write code

\usepackage{amsthm}
\usepackage{amsmath} % math. function by cases

\usepackage{amsfonts} % math fonts.

\usepackage{multicol}
\usepackage{multirow}

\usepackage[left=2.5cm, right=3cm]{geometry}


\title{Workshop \LaTeX}
\author{João Dionísio e Manuel Lamas}
\date{December 2022}

% Define onde ir buscar imagens
\graphicspath{ {./Images/} }

\newtheorem{theorem}{Theorem}

\newcommand{\N}{\mathbb{N}}
\newcommand{\Z}{\mathbb{Z}}
\newcommand{\Q}{\mathbb{Q}}
\newcommand{\R}{\mathbb{R}}


\begin{document}

\maketitle
    
\LaTeX\,é um sistema de software para edição de texto. Ao contrário daqueles a que estamos acostumados, existe uma separação entre o ficheiro onde escrevemos (\verb|.tex|) e o resultado final (\verb|.pdf|).

\section{Trying to simulate what it will be like}



Para aplicar enfâse a palavras usamos \textbf{bold} e \emph{itálico}, ou também \underline{sublinhado}~\footnote{É por isso que \LaTeX\,é mesmo fixe!}. Mas cuidado que usar sublinhado pode pode ter consequências \underline{graves} \underline{ok?}.~\cite{how_to_cite_stuff}

Para dar escape a carateres especiais. Usar backslash \verb|\|.
"10\% das pessoas"

Reparem que as áspas não foram colocadas corretamente. Para tal ``escrevemos assim''.

Podemos enumerar itens da seguinte maneira
\begin{enumerate}
    \item Primeiro
    \item Segundo
    \item Terceiro
\end{enumerate}

\begin{theorem}
    \LaTeX\,is super cool.
\end{theorem}
\begin{proof}
\LaTeX\, consegue fazer equações bonitas:
\begin{equation} % numbered equation
    e^{i\pi} = -1
\end{equation}
Mesmo que sejam complicadas:
\begin{equation} \label{eq:example}
    \forall \theta \in \Omega, \exists \iota \in \R~\mid \int_{-\infty}^{\infty+1}\sum_{\zeta=\mu}^{\alpha} \theta^{\iota} \, d\theta \geq \frac{\overline{\Xi}}{\overline{\Xi}} 
\end{equation}

\end{proof}

Pode dar muito jeito também referenciar equações que ficaram para trás. Isso é muito fácil em \LaTeX~\ref{eq:example}.

% É assim que se faz um comentário

\begin{table}[h]
    \centering
    \begin{tabular}{c|c}
         Pros & Cons \\    
        \hline
         Much prettier. $k \in \mathcal{K}$ vs. k € K & ???? \\ 
         Very easy to cite stuff~\cite{how_to_cite_stuff} & ????  \\
         cell7 & ????  \\ 
    \end{tabular}
    \caption{Prós e Contras de usar \LaTeX}
    \label{tab:my_label}
\end{table}



Para termos tabelas como a da tabela~\ref{tab:table_multi} usamos estas packages:
\begin{itemize}
    \item \verb|\usepackage{multicol}|
    \item \verb|\usepackage{multirow}|
\end{itemize}


\begin{table}[h!]
    \centering

    \begin{tabular}{ |c|c|c|c| }
    	\hline
    	\multirow{2}{*}{Test} & \multicolumn{3}{c|}{Note Code} \\
        % asdasd
    	\cline{2-4}
    	& A & B & C \\
    	\hline
    	-1 & 0 & 1 & 2 \\
    	\hline
    	0 & 12 & 13 & 14 \\
    	\hline
    	1 & 24 & 25 \\
    	\cline{1-3} % To only create the line between the specified columns
    \end{tabular}
    \caption{Testing a table}
    \label{tab:table_multi} % Must be after the caption
\end{table}


Para mais sobre matrizes ver \href{https://www.overleaf.com/learn/latex/Matrices}{Matrices}.
Escrever matrizes é essencialmente como tabelas mas num environment como \verb|bmatrix|
$$
\left[\begin{matrix}
1 & 2\\
3 & 4
\end{matrix}\right]
$$

$$
\begin{bmatrix}
x_1 & -y_1\\
ax_1 & y_1+b
\end{bmatrix}
+
\begin{bmatrix}
x_2 & -y_2\\
ax_2 & y_2+b
\end{bmatrix}
$$

$$
\begin{pmatrix}
k \\ 1 \\ 1\\
\end{pmatrix}
=
\alpha\times\begin{pmatrix}
1 \\ 0 \\ 1
\end{pmatrix}
+
\beta\times\begin{pmatrix}
0 \\ 0 \\ k
\end{pmatrix}
+
\gamma\times\begin{pmatrix}
0 \\ 1 \\ 0
\end{pmatrix}
$$

Determinantes
$$\begin{vmatrix}
1 & 2\\
3 & 4
\end{vmatrix} = -2$$
% Good job spotting it

\section{Podemos fazer mais secções}
\subsection{E até subsecções!}
\subsubsection{E mesmo subsubsecções!!!!} 

\begin{figure}[H]
    \centering
    \includegraphics[width=.2\linewidth]{ron_graham_juggling.jpeg}
    \caption{Uma imagem vale por um \href{https://www.youtube.com/watch?v=XTeJ64KD5cg}{número de Graham} de palavras}
    \label{fig:my_label}
\end{figure}



\begin{figure}[H]
\centering
\begin{subfigure}{.3\textwidth}
  \centering
  \includegraphics[width=.5\linewidth]{ron_graham_juggling.jpeg}
  \caption{Uma imagem vale por um \href{https://www.youtube.cuom/watch?v=XTeJ64KD5cg}{número de Graham} de palavras}
  \label{fig:sub1}
\end{subfigure}%
~
\begin{subfigure}{.3\textwidth}
  \centering
  \includegraphics[width=.5\linewidth]{homer_juggling.png}
  \caption{A subfigure}
  \label{fig:sub2}
\end{subfigure}
\caption{And then you can caption both}
\label{fig:test}
\end{figure}


\begin{figure}[H]
\centering
\begin{subfigure}{.3\textwidth}
  \centering
  \includegraphics[width=.5\linewidth]{ron_graham_juggling.jpeg}
  \caption{Uma imagem vale por um \href{https://www.youtube.com/watch?v=XTeJ64KD5cg}{número de Graham} de palavras}
  \label{fig:sub1}
\end{subfigure}%

\begin{subfigure}{.5\textwidth}
  \centering
  \includegraphics[width=.8\linewidth]{homer_juggling.png}
  \caption{A subfigure}
\label{fig:sub2}
\end{subfigure}
\caption[A figure with two subfigures vertically]{Uma descrição mais longa da figura}
\label{fig:test_vertical}
\end{figure}


Na figura~\ref{fig:test} temos duas subfiguras dispostas na horizontal, e na figura~\ref{fig:test_vertical} temos dispostas na vertical.


Também é fácil trabalhar com funções por ramos, vejam o exemplo abaixo da função associada à~\href{https://en.wikipedia.org/wiki/Collatz_conjecture}{Conjetura de Collatz} (e como bónus, vejam também como é fácil colocar URLs). 
\[   
f(n) = 
     \begin{cases}
       \dfrac{n}{2}, \text{if $n$ is even}\\
       3n+1, \text{if $n$ is odd}\\
     \end{cases}
\]



\paragraph*{Macros:}
Podem-se criar comandos para facilitar o uso daqueles que se usam com maior frequência. Por exemplo o comando \verb|\N| pode ser criado para replicar $\mathbb{N}$ (de \verb|\mathbb{N}|).

E então $\N$.

O macro é criado acrescentando a seguinte linha ao preâmbulo:\\
\verb|\newcommand{\N}{\mathbb{N}}|

% Este exemplo é mais ilustrativo, normalmente criariamos macros não cópias de um comando.
% Mas de comandos mais compridos e detalhados que deixassem de tornassem aborrecido o seu uso




\bibliographystyle{plainurl}
\bibliography{bibliography}

\end{document}
