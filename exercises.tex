\documentclass[10pt]{article}
\usepackage[utf8]{inputenc}

%\interfootnotelinepenalty=10000 %force footnote in the same page
\usepackage{eurosym} % Euro symbol

\usepackage{hyperref} % Hyperlinks
\usepackage{tikz} %

\usepackage[demo]{graphicx}
\usepackage{subcaption}
\usepackage{tabularx} 


\usepackage{float} % to force image placement

\usepackage{listings} % to write code

\usepackage{amsthm}
\usepackage{amsmath} % math. function by cases

\usepackage{amsfonts} % math fonts.

\usepackage{multicol}
\usepackage{multirow}

\usepackage[left=2.5cm, right=3cm]{geometry}


\title{\LaTeX\,Workshop Exercise Sheet}
% \author{João Dionísio e Manuel Lamas}
\date{December 2022}

% Define onde ir buscar imagens
\graphicspath{ {./Images/} }

\newtheorem{theorem}{Theorem}

\newcommand{\N}{\mathbb{N}}
\newcommand{\Z}{\mathbb{Z}}
\newcommand{\Q}{\mathbb{Q}}
\newcommand{\R}{\mathbb{R}}




\begin{document}

    \maketitle


    
    \noindent\textbf{Exercise 1:} Reproduce the following:
        A \textbf{number} is a mathematical object used to count, measure, \textit{and} label. The original examples are the natural numbers \underline{1, 2, 3, 4}, and so forth. Other categories of numbers include:
    \begin{itemize}
        \item Rational numbers: $\dfrac{1}{2}$
        \item Transcendental numbers: $\pi$
        \item Irrational numbers: $\sqrt{2}, \sqrt[3]{2}$ 
        \item Real numbers: $1$
        \item Complex numbers: $a+bi$
    \end{itemize}


    \vspace{1em}
    
    \noindent\textbf{Exercise 2:} Reproduce the following:

    $$(\forall \varepsilon > 0 ) \, (\exists \delta > 0) \, (\forall x \in \R) \, (0 < |x - p| < \delta \implies |f(x) - L| < \varepsilon)$$
    
    \vspace{1em}
    
    \noindent\textbf{Exercise 3:} Reproduce the following (the images can be found \href{https://www.amazon.com/Mathematics-Student-Teacher-Pullover-Hoodie/dp/B09784RL5C}{here} and \href{https://i.redd.it/1zfte27tn3q21.jpg}{here}):

    
    \begin{figure}[H]
        \centering
        \begin{subfigure}{.4\textwidth}
          \centering
          \includegraphics[width=.7\linewidth]{Images/latex_true.jpeg}
          \caption{A trivial statement}
          \label{fig:latex_true}
        \end{subfigure}%
        ~
        \begin{subfigure}{.4\textwidth}
          \centering
          \includegraphics[width=\linewidth]{Images/latex_meme.jpg}
          \caption{A funny meme}
          \label{fig:latex_meme}
        \end{subfigure}
        \caption{Truer words have never been spoken.}
        \label{fig:both_images}
    \end{figure}


    \noindent\textbf{Exercise 4:} Math expression % Porque é literalmente assim que LaTeX chama à coisa. 
    \begin{equation}
        \exists n \geq 3, x,y,z \in \N \mid x^{n} + y^{n} = z^{n}
    \end{equation}
    
    This result follows naturally from the argument in Figure~\ref{fig:latex_true}, and we leave it as an exercise to the reader.\checkmark


    \noindent\textbf{Exercise 5:} Matrices and Tables
    $$
    \left[\begin{matrix}
    \text{cos}(\theta) & -\text{sin}(\theta)\\
    -\text{sin}(\theta) & \text{cos}(\theta)
    \end{matrix}\right]
    $$

    $$
    \begin{pmatrix}
    k \\ 1 \\ 1\\
    \end{pmatrix}
    =
    \alpha\times\begin{pmatrix}
    1 \\ 0 \\ 1
    \end{pmatrix}
    +
    \beta\times\begin{pmatrix}
    0 \\ 0 \\ k
    \end{pmatrix}
    $$


    \begin{table}[]
        \centering
        \begin{tabular}{|c|c|c|c|c|}
              \hline
             $\times$ & \textbf{1} & \textbf{i} & \textbf{j} & \textbf{k} \\
              \hline
            \textbf{1} & 1 & i & j & k \\
              \hline
            \textbf{i} & i & -1 & -k & -j \\
              \hline
            \textbf{j} & j & -k & -1 & i \\
              \hline
            \textbf{k} & k & j & -i & -1 \\
              \hline
        \end{tabular}
        \caption{Quaterinon multiplication table}
        \label{tab:my_label}
    \end{table}

    \vspace{1em}

    \noindent\textbf{Exercise 6:} Prove that if $n^{2}$ is even, then $n$ is even.
    
    \vspace{1em}
    \noindent\textbf{Challenge:} Let $A$ be the set of solvable problems in polynomial time by a deterministic Turing machine and $B$ the set of solvable problems in polynomial time by a non-deterministic Turing machine. Is $A=B$?\\
    \textbf{Reward:} \euro$100$
\end{document}